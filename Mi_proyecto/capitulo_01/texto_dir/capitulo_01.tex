%%%%%%%%%%%%%%%%%%%%%%%%%%%%%%%%%%%%%%%%%%%%%%
%%%%%%%%%%%%%%%%%%%%%%%%%%%%%%%%%%%%%%%%%%%%%%
%%			     Capitulo 1 			    %%
%%%%%%%%%%%%%%%%%%%%%%%%%%%%%%%%%%%%%%%%%%%%%%
%%%%%%%%%%%%%%%%%%%%%%%%%%%%%%%%%%%%%%%%%%%%%%

\chapter{Introducción} \label{cap1}

Este primer capítulo trata de exponer una breve introducción sobre el problema propuesto, sus antecedentes históricos y los objetivos que se pretenden alcanzar durante el desarrollo de la investigación.

\newpage

%%%%%%%%%%%%%%%%%%%%%%%%%%%%%%%%%%%%%%%%%%%%%%%%%%%%%%%
%		            	Seccion          			  %
%%%%%%%%%%%%%%%%%%%%%%%%%%%%%%%%%%%%%%%%%%%%%%%%%%%%%%%
\section{Propuesta de investigación} \label{s1_1}

Empleados de una oficina, banco o empresa, profesores, alumnos, secretarios, etc. Todos ellos tienen en común que, a diario trabajan con uno o varios ordenadores ya sea para visualizar distintos documentos en cada monitor, realizar diversas tareas para, por ejemplo, agilizar los procesos de cálculo, etc. El proyecto planteado propone un sistema que ofrece la posibilidad de llevar a cabo varias tareas a la vez con diferentes equipos mediante el uso de los mismos dispositivos de control (ratón y teclado) a través de un mecanismo capaz de detectar la orientación del conjunto cabeza-ojos basándose en los movimientos comunes que realiza cualquier persona cuando centra su atención en una u otra pantalla con el propósito de facilitar el trabajo a este tipo de personas, lo cual, incentiva la tarea de investigación. De esta forma, el usuario no trabaja con la incomodidad  de cambiar una y otra vez de teclado y ratón permitiéndole ahorrar espacio en su escritorio debido a que el sistema implementado funciona interpretando estos cambios de posición tan intuitivos. En resumidas cuentas, el usuario dispone de un ratón y un teclado convencionales con los cuales puede controlar un computador u otro en función del lugar hacia donde dirija su mirada, lo cual, se consigue con un sistema de seguimiento ocular o como se llama comúnmente en inglés, ``eye tracking'' a través del uso de una cámara situada en la pantalla del equipo principal. El sistema se implementa con el objetivo de controlar dos equipos distintos, por tanto, existe una pantalla secundaria que se sitúa a uno de los dos lados. Sin embargo, surgen algunas dudas acerca de este planteamiento como, por ejemplo, si es verdaderamente posible crearlo ya que se desconoce la complejidad de crear un proyecto de tales características. Si se asume la posibilidad de llevarlo a cabo, aparecen otro tipo de cuestiones como qué tipo de hardware es es más ideal para este caso y si existe software libre que ayude a la implementación del código. Como es de esperar, en ingeniería no existen soluciones concretas a determinados problemas sin embargo, la pretensión del proyecto es llevarlo a cabo de la forma mas fiel posible a los objetivos marcados.

%%%%%%%%%%%%%%%%%%%%%%%%%%%%%%%%%%%%%%%%%%%%%%%%%%%%%%%
%		            	Seccion          			  %
%%%%%%%%%%%%%%%%%%%%%%%%%%%%%%%%%%%%%%%%%%%%%%%%%%%%%%%

\section{Antecedentes históricos} \label{s1_2}

Actualmente existe un amplio abanico de soluciones software y hardware para compartir teclado y ratón entre varios ordenadores. En primer lugar, se habla de ``software KVM'' (Keyboard Video Mouse) que tiene la finalidad de compartir un teclado y un ratón entre varios equipos sin emplear hardware adicional. Estos proyectos se han llevado a cabo por cuestiones de necesidad y hoy en día, se puede acceder a ellos a través de internet. Algunos de ellos son:

\begin{itemize}
    \item Synergy: Es un software cuyo desarrollador es Chris Schoeneman. Se trata de una herramienta que permite el uso compartido de un único ratón y teclado para manejar varios ordenadores desde un único escritorio situando el puntero del ratón en el monitor correspondiente. Es ``open source'' o de código libre bajo la licencia de GNU. Dado que no requiere hardware adicional, los equipos se controlan a través del uso de la red de área local. El equipo de Synergy decidió encriptar las conexiones y por tanto, inventaron su propio protocolo sin embargo, no utiliza ningún mecanismo de autentificación ni de cifrado. Además, se trata de un software de pago y es compatible con Windows, Linux y macOS.
    \item Input Director: El autor de este proyecto es Shane Richards. Este software tiene la misma utilidad que Synergy sin embargo, solo sirve para equipos con el sistema operativo de Windows. El control se realiza de la misma forma que Synergy o incluso con una combinación de teclas que permite cambiar de ordenador. Algunos usuarios prefieren este software frente al anterior debido a que la configuración de este último es más sencilla además de no generar problemas con esclavos que tienen un sistema operativo Windows 7 y de ser gratuito.
    \item Multiplicity: Pertenece a la compañía Stardock. La funcionalidad es la misma que los anteriores. Es de pago y su tarifa más barata solo permite el control de hasta dos equipos. Ambos ordenadores deben estar conectados en la misma red y únicamente puede instalarse en equipos con Windows 7, 8 y 10.
\end{itemize}

Existe otro tipo de software que ofrece prestaciones similares a los programas descritos, como Across, que utiliza como tecnología de conexión Bluetooth y ShareMouse siendo ambos compatibles con dispositivos Android.

En segundo lugar, existe hardware denominado ``hardware KVM'' (KVM switches) que conmuta a un equipo u a otro según una combinación específica de teclas. A estos dispositivos se les conecta los periféricos USB, los conectores VGA, audio y micrófono de los ordenadores que se controlan (entradas del switch) y las salidas se conectan al monitor, altavoces y micrófono, y equipo principales como se muestra en la figura~\ref{fig:switchKVM}.

\begin{figure}[h!]
\centering
\includegraphics[scale = 0.7]{capitulo_01/figuras_dir/KVM.jpg}
\caption{Control de equipos mediante un switch KVM. Imagen tomada de Topcomputer - Conmutador KVM ATEN CS52A-A7: https://topcomputer.ru/tovary/576596/}
\label{fig:switchKVM}
\end{figure}

En este caso, todos los equipos se controlan a través de una única pantalla. Si se desea usar un monitor por cada equipo independiente, no es necesario que el switch tenga entrada ``Video''. A este hardware se le llama ``hardware KM'' (figura \ref{fig:switchKM}).

\begin{figure}
\centering
\includegraphics[scale = 0.5]{capitulo_01/figuras_dir/KMSwitch.jpg}
\caption{Switch KM de dos puertos USB. Imagen tomada de Lindy – 2 Port USB KM Switch: https://www.lindy.co.uk/kvm-c6/kvm-switches-c295/2-port-usb-km-switch-p7901}
\label{fig:switchKM}
\end{figure}

Hay varios tipos de ``hardware KM'' en función del tipo y del número de conectores y de varias marcas y modelos.

A diferencia de todos los inventos existentes, este proyecto propone un control de equipos mediante el uso de hardware (microcontroladoras) y software (programa desarrollado en C) usando como medio de transmisión la red local vía inalámbrica y el protocolo propio de la red Ethernet (evitando conflictos como sucede con Synergy) incorporando un aspecto original como es el control por vista de los equipos mediante periféricos destinados a ese uso. Además, se eliminan problemas de incompatibilidades con el sistema operativo y con los propios dispositivos y la adición de ficheros de código complementario en los distintos equipos para conseguir la correcta comunicación entre el maestro y los esclavos como sucede con algunos de los softwares ya implementados.

%%%%%%%%%%%%%%%%%%%%%%%%%%%%%%%%%%%%%%%%%%%%%%%%%%%%%%%
%		            	Seccion          			  %
%%%%%%%%%%%%%%%%%%%%%%%%%%%%%%%%%%%%%%%%%%%%%%%%%%%%%%%
\section{Definición del problema} \label{s1_3}

Este proyecto engloba tanto hardware como software. En referencia al hardware, se necesita un microcontrolador central que sea capaz de comunicarse con los periféricos. Este dispositivo tiene la función de recoger aquellos datos que envían los dispositivos de control, llamados dispositivos HID o de interfaz humana, para posteriormente enviarlos al correspondiente equipo. A estos datos se les denomina ``eventos'' y constituyen tanto pulsaciones de teclas como movimientos del ratón. A este microcontrolador van conectados todos los periféricos, incluidos aquellos que se encargan del seguimiento ocular para obtener los eventos oculares que son usados posteriormente para decidir a qué equipo enviar los eventos HID. Para detectar posición de los ojos (eventos oculares) se necesita una cámara, que en principio, no reúne ninguna característica especial.

Este microcontrolador se comunica con ambos equipos. La comunicación se puede establecer via SPI, I2C, Bluetooth e internet (Wifi o Ethernet), sin embargo, se opta por internet dado que su protocolo es uno de los más conocidos. La forma más sencilla de enviar y recibir datos desde el microcontrolador central es transmitiendo datos a otro microcontrolador que esté conectado a cada uno de los equipos. Dicho lo anterior, el esquema de la figura \ref{fig:esquemahardware} muestra la relación entre los distintos dispositivos involucrados en el proyecto.

\begin{figure}
\centering
\includegraphics[scale = 0.6, angle=-90]{capitulo_01/figuras_dir/esquemahardware.jpg}
\caption{Esquema del hardware empleado}
\label{fig:esquemahardware}
\end{figure}


Para llevar a cabo lo descrito anteriormente, se necesita implementar la parte software del proyecto. Esta parte engloba un conjunto de programas dedicados a determinadas tareas. En la sección \ref{s1_4} a cada una de estas tareas se le asigna un objetivo.

%%%%%%%%%%%%%%%%%%%%% Subsección %%%%%%%%%%%%%%%%%%%%%
\subsection{Estado del arte} \label{s1_3_1}

Para comenzar a desarrollar el proyecto, es necesario, en primer lugar, disponer del hardware adecuado. Este proyecto tiene como objetivo manejar dispositivos periféricos, enviar y recibir datos. Estas funciones las desempeñan fácilmente los microcontroladores.
Los microcontroladores son circuitos integrados cuya función consiste en controlar dispositivos de entrada y salida. Están formados por una memoria FLASH para almacenar el programa, la memoria de datos (RAM), la memoria de configuración o EEPROM, puertos analógicos, digitales y seriales y un procesador. Los microcontroladores son programables, y por ello, pueden usarse en innumerables aplicaciones. La elaboración de programas se lleva a cabo mediante lenguaje de alto nivel, en este caso, C y C++, que posteriormente se convierte en instrucciones fácilmente interpretables por el microcontrolador mediante el código máquina. De esta transformación se encarga el compilador. Una vez transformado, se almacena en la memoria ROM mediante un programador.

Existen varios fabricantes de microcontroladores como Atmel, Microchip Technology, Toshiba, Intel... Sin embargo, la elección del microcontrolador depende fundamentalmente de la aplicación. Ya que hay varios microcontroladores en el mercado con los que se puede implementar la misma aplicación, hay que tener en cuenta algunos aspectos antes de elegir definitivamente el microcontrolador que se pretende usar. En primer lugar, es necesario atender a sus características internas como su velocidad, memoria, número de pines de entrada y salida, etc. En segundo lugar, es importante saber si existe en la web una comunidad de desarrollo, herramientas de desarrollo gratuitas así como información variada aportada por el propio fabricante. También es necesario conocer si el microcontrolador está disponible y es de fácil acceso en el mercado, ya que puede ser difícil conseguirlo en caso de que las existencias sean limitadas o se venda en sitios muy concretos. Por último, es evidente que el precio importa. Para desarrollar una misma aplicación, el proyecto se puede encarecer si se elige el microcontrolador más caro.

Existen innumerables combinaciones de dispositivos con los que se podría llegar al objetivo final del proyecto, sin embargo, se puede realizar la selección de dispositivos dando prioridad a los aspectos mencionados. La mayoría de los dispositivos que se destacan a continuación son modelos de Arduino o Raspberry Pi debido a que cuentan con una gran comunidad de desarrollo, lo cual facilita enormemente la realización del proyecto, además de ser dispositivos económicos cuyas características permiten implementar el sistema de control propuesto. Una de las características internas más importantes a tener en cuenta es la conexión que se efectúa entre ellos para conocer la disponibilidad de puertos de conexión por si se presenta la ocasión en que hay que utilizar módulos adicionales. Los microcontroladores pueden conectarse entre sí de varias formas permitiendo que uno haga la función de maestro y el resto de esclavos, aunque puede haber excepciones en cuanto al número de maestros. La conexión entre los microcontroladores puede realizarse mediante USB, Bluetooth, SPI, I2C, o mediante red de área local (Ethernet, WiFi). Debido a las razones expuestas en la sección 2.3.1, los microcontroladores se conectan vía internet por el protocolo de transmisión de datos TCP. Esto implica que los microcontroladores secundarios dispongan de una interfaz Ethernet o Wifi. En caso contrario, sería necesario usar un módulo de conexión. Además, éstos deben ser capaces de implementar un mecanismo de simulación de teclas, es decir, deben ser reconocidos como dispositivos HID, y para ello, según lo mencionado en la sección 2.5, deben disponer de un microcontrolador especial. Esta característica la integran los modelos Leonardo de Arduino. Como Arduino Leonardo no dispone de interfaz Ethernet ni Wifi, es necesario usar un módulo. En este caso, debido a la experiencia de trabajo con el módulo Ethernet en cursos académicos anteriores, a su disponibilidad en el mercado y a su precio, se opta por este.

En la actualidad, el ratón y el teclado requieren ser conectados mediante USB. Por tanto, el microcontrolador maestro debe disponer de puertos USB. Algunas tarjetas Arduino como “Arduino TRE” y “Arduino Yun” y otras tarjetas Raspberry Pi como el modelo antiguo “Raspberry Pi 1 A” traen incorporado un solo conector USB. Esta característica podría ser válida en el caso de disponer de un teclado y un ratón inalámbricos con un único módulo USB. Si se dispone de los periféricos por separado, esta solución no es la más recomendable debido a que se necesita una tarjeta complementaria o "shield" como por ejemplo “Arduino USB Host Shield” o similar para aumentar el número de conectores USB. Debido a que el microcontrolador principal se conecta a tres periféricos distintos, son necesarios mínimo dos interfaces USB y otra correspondiente a la conexión de la cámara en el caso de usar un conector específico. Dado que los sistemas operativos basados en Linux tienen un sistema de archivos el cual permite obtener fácilmente los eventos de dispositivos HID, es ideal usar el modelo B de Raspberry Pi 3. Este microcontrolador cumple las características necesarias para implementar el sistema propuesto en cuanto a puertos de conexión e interfaz Ethernet y Wifi, es económico, dispone de una amplia comunidad de desarrollo, existe bastante información en la web acerca de él y al igual que con Arduino Leonardo, se puede trabajar con código abierto, es decir, código distribuido libremente de fácil acceso y con la posibilidad de modificarlo sin restricciones.


%%%%%%%%%%%%%%%%%%%%%%%%%%%%%%%%%%%%%%%%%%%%%%%%%%%%%%%
%			Seccion				                      %
%%%%%%%%%%%%%%%%%%%%%%%%%%%%%%%%%%%%%%%%%%%%%%%%%%%%%%%
\section{Objetivos del proyecto y resultados esperados} \label{s1_4}

El objetivo principal del proyecto consiste en realizar el prototipo del sistema basado en el hardware descrito en la sección 1.3 y el software implementado a lo largo de la investigación que permita conseguir el control de dos equipos distintos con los mismos dispositivos de interfaz humana mediante un mecanismo de seguimiento ocular tal y como se menciona en la sección 1.1. 
El objetivo principal puede conseguirse siempre y cuando se logre implementar los programas correspondientes a cada una de estas etapas que compone la tarea de investigación que constituyen los objetivos secundarios del proyecto:
\begin{itemize}
    \item Obtención y demultiplexado de los eventos del teclado y del ratón.
    \item Corrección de la distorsión radial de la cámara.
    \item Obtención de los eventos oculares.
    \item Seleccionar la dirección IP de Arduino Leonardo.
    \item Envío de los eventos de dispositivos HID a los microcontroladores secundarios (Arduino Leonardo).
    \item Recepción de los eventos de dispositivos HID por Arduino Leonardo.
    \item Simulación de los eventos recibidos como pulsaciones de teclas y movimientos del ratón.

\end{itemize}

Una vez logrados estos objetivos, se espera como resultado que el prototipo funcione tal y como describe el objetivo principal tras combinar el hardware escogido con el software implementado.

%%%%%%%%%%%%%%%%%%%%%%%%%%%%%%%%%%%%%%%%%%%%%%%%%%%%%%%
%			Seccion				                      %
%%%%%%%%%%%%%%%%%%%%%%%%%%%%%%%%%%%%%%%%%%%%%%%%%%%%%%%
\section{Metodología de la investigación y organización de la memoria} \label{s1_5}

La consecución de los objetivos fijados es posible bajo la dirección de Francisco Moya Fernández, quien ha facilitado documentación de ayuda mediante su artículo ``Taller de Raspberry Pi''. Con él, pretende familiarizar a los alumnos con todo aquello relacionado con el entorno de Linux. El libro consta de varias secciones que tratan diversos temas de interés. Una de las secciones más relevantes en el proyecto trata sobre la biblioteca ``Reactor'' que ha sido fundamental en el desarrollo del programa de lectura y demultiplexado de eventos de dispositivos HID haciendo uso de manejadores de tipo ``event handler'' como se describe en secciones posteriores de este documento. El programa del taller de Raspberry Pi no sólo incluye el libro anterior sino que presta un microcontrolador Raspberry Pi para el desarrollo de software en el entorno de Raspbian, un sistema operativo basado en Debian. El libro del taller recomienda además el uso de un sistema de control de versiones o GIT, que mantiene un registro de todas las modificaciones que se realizan sobre un programa tras el ``comiteado'' del archivo\footnote{Para más información, véase el libro Git Pro \citep{GIT}}. De esta forma, se evitan problemas por pérdida de información debida a defectos en sectores del disco duro o incluso la rotura total del mismo ya que todas las versiones se almacenan en la nube.

Junto a las explicaciones y bibliografía aportadas por el director del proyecto, la comunidad de desarrollo de software libre ayuda enormemente la tarea de investigación. Además, se ha establecido como metodología de planificación la llamada ``planificación ágil''. Consiste en conocer los requisitos del cliente y en base a ello, crear las llamadas ``historias de usuario'' que constituyen las tareas a realizar para satisfacer los requisitos que el cliente impone sobre el producto. El director del proyecto actúa como cliente, decidiendo qué requisitos son los más importantes y entre ambos, se realiza una estimación del tiempo a invertir en cada historia. Posteriormente, en función de la prioridad que otorgue el director a las historias de usuario, se agrupan formando iteraciones de duración total entre una y dos semanas. El resultado de esas iteraciones debe ser algo claramente apreciable por el cliente.

Respecto a la memoria, se divide principalmente en cuatro bloques: conceptos teóricos, procedimientos, resultados y conclusiones. A su vez, los tres primeros están formados por las secciones relativas a los objetivos secundarios, ya que la realización de todos ellos supone la finalización del proyecto.

